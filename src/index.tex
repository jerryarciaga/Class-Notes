\section {Assess Organizational Security with Network Reconnaisance Tools}

\section {Explain Security Concerns with General Vulnerability Types}
	\subsection {Software Vulnerabilities and Patch Management}
		\begin{itemize}
			\item Exploits for faults in software code
			\item Applications
			\item Operating System
			\item Firmware \\
				-- PC Firmware \\
				-- Network Appliances and IOT Devices
			\item Improper or Weak Patch Management \\
				-- Undocumented Assets \\
				-- Failed updates and removed patches
		\end{itemize}
	\subsection {Zero-Day and Legacy Platform Vulnerabilities}
		\subsubsection {Zero-Day}
			\begin{itemize}
				\item Vulnerability is unknown to the vendor
				\item Threat actor develops an exploit for which there is no patch
				\item Likely to be used against high value targets
			\end{itemize}
		\subsubsection {Legacy Platform}
			\begin{itemize}
				\item Vendor no longer releases security patches
			\end{itemize}
	\subsection {Weak Host Configurations}
		\begin{itemize}
			\item Default Settings
			\item Unsecured Root Accounts
			\item Open Permissions
		\end{itemize}
	\subsection {Weak Network Configurations}
		\subsubsection {Open ports and services}
			\begin{itemize}
				\item Restrict using an \textbf{Access Control List}
				\item Disable unnecessary services or block ports
				\item Block at network perimeter
			\end{itemize}
		\subsubsection {Unsecure Protocols}
			Cleartext data transmissions are vulnerable to snooping
		\subsubsection {Weak Encryption}
			\begin{itemize}
				\item Storage and transport encryption
				\item Key is generated from a weak password
				\item Cipher has weaknesses
				\item Key distribution is not secure
			\end{itemize}
		\subsubsection {Error Messages}
			Error messages reveal too much information
	\subsection {Impacts from Vulnerabilites}
		\begin{itemize}
			\item Data breaches and data exfiltration impacts
			\item Identity theft
			\item Data loss and availability loss impacts
			\item Financial and reputation impacts
		\end{itemize}
	\subsection {Third-Party Risks}
		\begin{itemize}
			\item Supply chains \\
				-- Due diligence \\
				-- Weak links
			\item Vendor Management
			\item Outsourced Code Development
			\item Data storage
			\item Cloud-based vs on-premises risks
		\end{itemize}

\section {Summarize Vulnerability Scanning Techniques}
	\subsection {Security Assessment Frameworks}
		\begin{itemize}
			\item Methodology and scope for security assessments
			\item NIST SP 800-115 -- Testing, examining, interviewing
			\item Vulnerability assessment versus threat hunting and penetration testing
			\item Vulnerability assessments can use a mix of \textbf{manual procedues}
				and \textbf{automated scanning tools}
		\end{itemize}
	\subsection {Vulnerability Scan Types}
		\begin{itemize}
			\item Automated scanners configured with list of known vulnerabilities
			\item Network vulnerability scanner
			\item Application and web application scanners
		\end{itemize}
	\subsection {Common Vulnerabilities and Exposures}
		\begin{itemize}
			\item Vulnerability Feed/Plug-in/Test
			\item Security Content Automation Protocol (SCAP)
			\item Common Vulnerabilities and Exposures (CVE)
			\item Common Vulnerability Scoring System (CVSS)
		\end{itemize}
	\subsection {Intrusive vs Non-Intrusive Scanning}
		\begin{itemize}
			\item Remote scanning vs Agent-based Scanning
			\item Non-intrusive scanning \\
				-- Passively test security controls \\
				-- Scanners attach to network and only sniff traffic \\
				-- Possibly some low-interaction with hosts
			\item Intrusive/Active scanning \\
				-- Establish network session \\
				-- Agent-based scan
		\end{itemize}
	\subsection {Credentialed vs Non-credentialed Scanning}
		\subsubsection {Non-credentialed}
			\begin{itemize}
				\item Anonymous or guest access only
				\item Might test default passwords
			\end{itemize}
		\subsubsection {Credentialed}
			\begin{itemize}
				\item Scan configured with logon
				\item Can allow privileged access to configuration settings/logs/registry
				\item Use dedicated account for scanning
			\end{itemize}
	\subsection {Configuration Review}
		\begin{itemize}
			\item Lack of controls -- Security controls that should be present but are not
			\item Misconfiguration -- Settings deviate from template configuration
			\item Driven by templates of configuration settings
			\item Compliance-based templates available in many products
		\end{itemize}
	\subsection {Threat Hunting}
		\begin{itemize}
			\item Use log and threat data to search for IOCs
			\item Advisories and bulletins
			\item Intelligence fusion and threat data
			\item Maneuver
		\end{itemize}

\section {Explain Penetration Testing Concepts}
	\subsection {Rules of Engagement}
		\begin{itemize}
			\item Agreement for objectives and scope
			\item Authorization to proceed from system owner and affected third-parties
			\item Attack profile \\
				-- \textbf{Black box} (unknown environment) \\
				-- \textbf{White Box} (known environment) \\
				-- \textbf{Grey Box} (partially known)
			\item Bug bounty programs
		\end{itemize}
	\subsection {Exercise Types}
		\begin{itemize}
			\item \textbf{Red Team} -- offensive role
			\item \textbf{Blue Team} -- defensive role
			\item \textbf{White Team} -- Sets the rules of engagement and monitors
				the exercise
			\item \textbf{Purple Team} -- red and blue share info and debrief regularly
		\end{itemize}
	\subsection {Pen Test Attack Life Cycle}
		\subsubsection {Attack Life Cycle}
			\begin{enumerate}
				\item Initial exploitation
				\item Persistence
				\item Privilege escalation
				\item Lateral movement
				\item Pivoting
				\item Actions on
				\item Cleanup
			\end{enumerate}
		\subsubsection {Penetration Testing Life Cycle}
			\begin{enumerate}
				\item Information Gathering
				\item Threat Modeling
				\item Vulnerability Analysis
				\item Exploitation
				\item Post Exploitation
				\item Reporting
			\end{enumerate}

\section {Identifying Social Engineering and Malware}
	\subsection {Compare and Contrast Social Engineering Techniques}
		\subsubsection {Social Engineering}
			\begin{itemize}
				\item "Hacking the Human"
				\item Purposes of Social Engineering \\
					-- Reconnaissance and eliciting information \\
					-- Intrusion and gaining unauthorized access
				\item Many Possible Scenarios \\
				 	-- Persuade a user to run a malicious file \\
					-- Contact a help desk and solicit information \\
					-- Gain access to premises and install a monitoring device
			\end{itemize}
		\subsubsection {Reasons for Effectiveness}
			\begin{itemize}
				\item Familiarity/Liking -- Establish trust
				\item Consensus/Social Proof -- Exploit polite behaviors
				\item Authority and Intimidation -- Make target afraid to refuse
				\item Scarcity and Urgency -- Rush the target into a decision
			\end{itemize}
		\subsubsection {Impersonation and Trust}
			\begin{itemize}
				\item Impersonation -- Pretend to be someone else
				\item Pretexting -- Using a scenario with convincing additional detail
				\item Trust -- Obtain and spoof data
			\end{itemize}
		\subsubsection {Dumpster Diving and Tailgating}
			\begin{itemize}
				\item Dumpster diving -- Steal documents and media from trash
				\item Tailgating -- Access premises directly
				\item Piggy backing -- Access premises without authorization
			\end{itemize}
		\subsubsection {Identity Fraud and Invoice Scams}
			\begin{itemize}
				\item Identity fraud -- Impersonation with convincing detail
				\item Invoice scams -- Spoofing supplier details to submit invoices
				\item Credential theft -- Credential Harvesting, shoulder surfing
			\end{itemize}
		\subsubsection {Phishing, Whaling, and Vishing}
			\begin{itemize}
				\item Trick target into using a malicious resource
				\item Spoof legitimate communications
				\item Spear phishing -- Highly targeted/tailored attack
				\item Whaling -- Targets senior management
				\item Vishing -- Using a voice channel
				\item SMishing -- Using text messaging
			\end{itemize}
		\subsubsection {Spam, Hoaxes and Prepending}
			\begin{itemize}
				\item Spam -- unsolicited email, spam over instant messaging (SPiM)
				\item Hoaxes -- Delivered as spam or malvertising, get user to install rdp
				\item Prepending -- Tagging email subject line, warn users
			\end{itemize}
		\subsubsection {Pharming and Credential Harvesting}
			\begin{itemize}
				\item Passive Techniques have less risk of detection
				\item Pharming -- DNS spoofing
				\item Typosquatting -- Use cousin domains instead of redirection
				\item Watering Hole -- Target a third party site
				\item Credential Harvesting -- Attacks focused on obtaining credentials
					for sale
			\end{itemize}
		\subsubsection {Influence Campaigns}
			\begin{itemize}
				\item Sophisticated threat actors use multiple resources to change opinions
				\item Soft power -- Leveraging diplomatic and cultural assets
				\item Hybrid warfare -- Use of espionage, disinformation, and hacking
				\item Social media -- Use of hacked accounts and bot accounts
			\end{itemize}
	\subsection {Analyze Indicators of Malware-based Attacks}
		\subsubsection {Malware Classification}
			\begin{itemize}
				\item Classification of vector or infection method
				\item Viruses and worms -- spread code without authorization
				\item Trojans -- malicious program concealed within a benign one
				\item Potentially unwanted programs/applications (PUPs/PAPs) \\
					-- Pre-installed bloatware or installed alongside another app \\
					-- Installation may be covert \\
					-- Also called grayware
				\item Classification by payload	
			\end{itemize}
		\subsubsection {Computer Viruses}
			\begin{itemize}
				\item Rely on some sort of host file or media
				\item Multipartite
				\item Polymorphic
				\item Vector for delivery
			\end{itemize}
		\subsubsection {Computer worms and Fileless malware}
			\begin{itemize}
				\item Early computer worms -- Propagate in memory over network links
				\item Fileless malware -- Exploiting RCE and memory residence, shellcodes
				\item Advanced Persistent Threats(APT)/Advanced Volatile Threat(AVT)/
					Low Observable Characteristics(LOC)
			\end{itemize}
		\subsubsection {Backdoors and Remote Access Trojans}
			\begin{itemize}
				\item Backdoor malware
				\item Remote access trojans (RATs)
				\item Bots and Trojans
				\item Command and Control (C2)
				\item Backdoors from misconfiguration and unauthorized software
			\end{itemize}
		\subsubsection {Rootkits}
			\begin{itemize}
				\item Local administrator vs System/root privileges
				\item Replace key system files and utilities
				\item Purge log files
				\item Firmware rootkits
			\end{itemize}
		\subsubsection {Ransomware, Crypto-Malware, and Logic Bombs}
			\begin{itemize}
				\item Ransomware -- Nuisance (lock out user by replacing shell)
				\item Crypto-malware -- High impact ransomware (encrypt data files)
				\item Cryptomining/cryptojacking) -- Hijack resources to mine crypto
				\item Logic bombs
			\end{itemize}
		\subsubsection {Malware indicators}
			\begin{itemize}
				\item Browser changes or overt ransomware notification
				\item Anti-virus notifications -- Behavior based analysis
				\item Sandbox execution -- Cuckoo
				\item Resource utilization/consumption -- Task manager and top
				\item File system changes -- registry, temp files
			\end{itemize}
		\subsubsection {Process Analysis}
			\begin{itemize}
				\item Signature-based detection is failing to identify modern APT
				\item Network and host behavior anomalies and drive detection methods
				\item Running process analysis -- Process explorer
				\item Logging activity -- System Monitor
				\item Network Activity
			\end{itemize}

\section {Summarizing Basic Cryptographic Concepts}
	\subsection {Cryptographic Concepts}
		\begin{itemize}
			\item Encryption and Decryption -- encoding and decoding \\
				-- Plaintext is the decoded message \\
				-- Ciphertext is the coded message \\
				-- Cipher is the means of change of algorithm \\
				-- Cryptanalysis is the art of cracking crypto systems
			\item Alice(Sender), Bob(Receiver), Mallory(Intruder)
		\end{itemize}
	\subsection {Hashing Algorithms}
		\begin{itemize}
			\item Fixed length hash from variable string with cryptographic properties
			\item Used for password storage and checksums(integrity)
			\item Secure Hashing Algorithms(SHA)
			\item Message Digest Algorithms(MD5)
		\end{itemize}
		% TODO Put Hashing Table Here
	\subsection {Encryption Ciphers and Keys}
		\begin{itemize}
			\item Hashing is not encryption --- the process is not reversible
			\item Encryption uses a reversible process based on a secret
			\item Process should be too complex to unravel without a secret
			\item Cannot keep the cipher/algorithm itself secret
			\item Key ensures ciphertext remains protected
			\item Protecting the key is easier than protecting the algorithm
		\end{itemize}
	\subsection {Symmetric Encryption}
		\begin{itemize}
			\item Same secret key is used for encryption and decryption
			\item Fast -- suitable for bulk encryption of large amounts of data
			\item Problem storing and distributing key securely
			\item Confidentiality only -- sender and recipient know the same key
		\end{itemize}
	\subsection {Stream and Block Ciphers}
		\begin{itemize}
			\item Stream ciphers -- decrypt/encrypt bit/byte at a time \\
				-- Must be used with an initialization vector (IV)
			\item Block ciphers -- Treat data as equal blocks, using padding as necessary
			\item Key length \\
				-- Range of key values is the keyspace \\
				-- Longer key bit means larger keyspace \\
				-- Strength of key of given length varies between ciphers
		\end{itemize}
	\subsection {Asymmetric Encryption}
		\begin{itemize}
			\item Public/private key pair \\
				-- If the public key encrypts, only the private key can decrypt \\
				-- If the private key encrypts, only the public key can decrypt \\
				-- Public key cannot be derived from the private key \\
				-- \textbf{Private key} must be kept secret \\
				-- \textbf{Public key} is easy to distribute
			\item Message size is limited to key size so not suitable for large amounts
				of data
			\item Used for small amounts of authentication data
		\end{itemize}
	\subsection {Public Key Algorithms}
		\begin{itemize}
			\item RSA algorithm (Rivest, Shamir, Adleman) \\
				-- Basis of many public key cryptography systems \\
				-- Trapdoor function
			\item Elliptical Curve
		\end{itemize}
	\subsection {Summarize Cryptographic Use Cases and Weaknesses}
		\subsubsection {Digital Signatures}
			\begin{itemize}
				\item Using public key for hashing
				\item Digital signature provide integrity, authentication and
					non-repudiation
				\item RSA-based digital signatures
				\item Digital Signature Algorithm (DSA) with ECC Cipher
			\end{itemize}
		\subsubsection {Digital Envelopes and Signatures}
			\begin{enumerate}
				\item Alice obtains a copy of Bob's public key
				\item Alice encrypts a file using a symmetric key
				\item Alice encrypts the symmetric key using Bob's public key
				\item Alice sends the ciphertext and encrypted symmetric key to Bob
				\item Bob decrypts the symmetric key with his symmetric key
				\item Bob decrypts the ciphertext with symmetric key
			\end{enumerate}
		\subsubsection {Digital Certificates}
			\begin{itemize}
				\item Wrapper for a public key to associate with a digital identity
				\item Both parties must trust the CA (Certificate Authority)
			\end{itemize}
		\subsubsection {Perfect Forward Secrecry}
			\begin{itemize}
				\item RSA Key decrypts the session key using the server private key
				\item The private key stored on the server may be compromised in the future
			\end{itemize}
		\subsubsection {Authenticated Modes of Operation}
			\begin{itemize}	
				\item Unauthenticated encryption
				\item Authenticated encryption \\
					-- \textbf{Message authentication code} -- provides authentication and
						integrity \\
					-- Uses AES CBC with HMAC-SHA
				\item Authenticated encryption with Additional Data (AEAD)
			\end{itemize}
		\subsubsection{Cryptography Supporting Confidentiality}
			\begin{itemize}
				\item Hybrid encryption
				\item File encryption
				\item Transport encryption
			\end{itemize}

\section {Implementing Public Key Infrastructure}
	\subsection {Private and Public Key Usage}
		\begin{itemize}
			\item Public Key Cryptography \\
				-- \textbf{Public Key} encrypts the message \\
				-- \textbf{Private Keys} decrypt and authenticate the message
		\end{itemize}
	\subsection {Certificate Authorities}
	\subsection {PKI Trust Models and Certificate Chaining}
		\begin{itemize}
			\item Single CA
			\item Hierarchical / Chain of trust -- Root CA, Intermediate CAs, Leaf
			\item Online vs Offline
		\end{itemize}
	\subsection {Registration and CSRs}
		\begin{itemize}
			\item Registration identification and authentication procedures
			\item Certificate Signing Request (CSR) \\
				-- Client generates key pair and sends public key to CA with CSR \\
				-- CA performs subject identity checks \\
				-- CA signs and issues certificate
			\item Registration Authority  (RA)
		\end{itemize}
	\subsection {Digital Certificates}
	\subsection {Types of Cetrtificates}
		\begin{itemize}
			\item Certificate policies and templates
			\item Key usage
			\item Extended key/Enhanced Key Usage
			\item Critical or Non-Critical
		\end{itemize}
	\subsection {Web Server Certificate Types}
		\begin{itemize}
			\item Domain validation (DV) -- more rigorous identity checks
			\item Extended Validation (EV) -- even more rigorous identity checks \\
				-- They do not allow domains with wildcards
		\end{itemize}
	\subsection {Other Certificate Types}
		\begin{itemize}
			\item Machine/computer
			\item Email/user certificate -- identify by email address
			\item Code signing -- validate publisher name
			\item Root certificate -- self-signed for CA
			\item Self-signed certificate -- must by manually trusted
		\end{itemize}

\section {Implement PKI Management}
	\subsection {Key Recovery and Escrow}
		\begin{itemize}
			\item \textbf{M-of-N} for critical keys(root servers)
			\item Keys can be backed up to protect against data loss
			\item Escrow backup -- placing archived keys with a trusted third party
		\end{itemize}
	\subsection {Certificate Expiration}
		\begin{itemize}
			\item Certificate duration
			\item Certificate renewal -- use existing key pair, re-key with generated pair
			\item Expiration -- public key no longer accepted, archive/destroy
		\end{itemize}
	\subsection {Certificate Revocation Lists}
		\begin{itemize}
			\item Revocation vs suspension
			\item Reason codes
			\item Certificate Revocation List(CRL) -- lists revoked and suspended \\
				-- Browser CRL Checking
		\end{itemize}
	\subsection {Online Certificate Status Protocol Responders}
		\begin{itemize}
			\item Online Certificate Status Protocol -- Client queries single cert
			\item OCSP Stapling
		\end{itemize}
	\subsection {Certificate Pinning}
	\subsection {Certificate Formats}
		\begin{itemize}
			\item Distinguished Encoding Rules (DER) -- Binary Format (Windows)
			\item Privacy-enhanced Electrnonic Mail (PEM)
			\item .CER (Windows and Linux)  and .CRT(Linux) file formats
			\item Personal information exchange
			\item Export a certificate chain
		\end{itemize}
	\subsection {OpenSSL}
		\begin{itemize}
			\item Windows Certificate Services and \texttt{certutil}/Powershell
			\item OpenSSL \\
				-- Key pair generation and CA root certificate \\
				-- Certificate requests \\
				-- Viewing and verifying certificates \\
		\end{itemize}
	\subsection{Certificate Issues}
		\begin{itemize}
			\item Troubleshoot rejection of certificates by servers and clients
			\item Audit certificate and PKI infrastracture
		\end{itemize}

\section {Implementing Authentication Controls}
	\subsection {Identity and Access Management}
		\begin{itemize}
			\item \textbf{Subjects} -- users/software that request access
			\item \textbf{Objects} -- resources such as networks, servers, data
			\item \textbf{Identification} -- subject + computer network account
			\item \textbf{Authentication} -- Challenge to subject
			\item \textbf{Authorization} -- rights and permissions assigned
			\item \textbf{Accounting} -- auditing use of the account
			\item AAA Services -- Authentication, Authorization, Accounting
		\end{itemize}
	\subsection {Authentication Factors}
		\begin{itemize}
			\item Something you know -- password, pin, challenge questions
			\item Something you have -- ownership factor, hardware tokens, 2fa
			\item Something you are -- biometric factor
			\item It's important to have multiple forms of these
		\end{itemize}
	\subsection {Authentication Design}
		\begin{itemize}
			\item Meet requirements for CIA triad
			\item Confidentiality -- keep credentials secure
			\item Integrity -- threat actors cannot bypass or subvert auth mechanism
			\item Availability -- does not cause undue delay or support issues 
				(99.99% uptime)
		\end{itemize}
	\subsection {Multifactor Authentication}
		\begin{itemize}
			\item Strong authentication requires two or three types
				-- Knowledge factor is weak in terms of confidentiality
			\item Multifactor Authentication (MFA)
			\item Two-Factor Authentication (2FA) -- must be two \textbf{different}
				factors
		\end{itemize}
	\subsection {Authentication Attributes}
		\begin{itemize}
			\item Somewhere you are -- geolocation, IP location, geofencing
			\item Something you can do -- unique action patterns like the way you
				hold your phone
			\item Something you can exhibit -- a behavior or personality trait
			\item Someone you know -- web of trust, you have to know another individual
		\end{itemize}

\section {Implement Knowledge-Based Authentication}
	\subsection {Local, Network and Remote Authentication}
		\begin{itemize}
			\item Authentication Providers -- passwords vs password hashes
			\item Windows authentication -- local sign-in, network(Kerberos), remote
			\item Linux authentication -- /etc/passwd and /etc/shadow,
				pluggable authentication modules (PAMs)
			\item Single Sign-On(SSO)
		\end{itemize}
	\subsection {Kerberos Authentication}
		\begin{itemize}
			\item SSO and authentication provider
			\item Clients
			\item Application Servers
			\item Key Distribution Center(KDC) \\
				-- Authentication service -- Ticket Granting Ticket \\
				-- Ticket Granting Service -- Service Ticket
		\end{itemize}
	\subsection {PAP, CHAP, MS-CHAP Authentication}
		\begin{itemize}
			\item Password Authentication Protocol -- unsecure unless under encrypted
				tunnel
			\item Challenge Handshake Authentication Protocol (CHAP) -- similar to NTLM \\
				-- repeated during the session to prevent replay attacks \\
				-- various implementations \\
				-- Not secure enough to use without encrypted tunnel
		\end{itemize}
	\subsection {Password Attacks}
		\begin{itemize}
			\item Plaintext/unencrypted -- sniffing from unsecure controls/repos
			\item Online password attacks -- interaction with authentication service
			\item Horizontal brute forcing/password spraying
			\item Offline attacks \\
				-- Password database \\
				-- Hash transmitted directly \\
				-- Hash used as key to sign as HMAC
		\end{itemize}
	\subsection {Brute force and Dictionary Attacks}
		\begin{itemize}
			\item Exploit weak user/pass combinations and mechanisms
			\item Brute force attack
			\item Dictionary attack -- rainbow tables, salt
			\item Hybrid attack -- dictionary + bruteforce, fuzzing of dictionary terms
		\end{itemize}
	\subsection {Authentication Management}
		\begin{itemize}
			\item Hardware and software for storing and submitting multiple user passwords
			\item Password key -- USB token, bluetooth/NFC
		\end{itemize}

\section {Implementing Authentication Technologies}
	\subsection {Smart Card Authentication}
		\begin{itemize}
			\item Kerberos-based smart card logon
			\item Card readers
		\end{itemize}
	\subsection {Key Management Devices}
		\begin{itemize}
			\item Provision keys with insider threat risk reduced
			\item Smart cards and usb keys
			\item Trusted platform module (TPM) -- virtual smartcards
			\item Hardware Security Module (HSM) \\
			 -- Provision keys across the network \\
			 -- Key archive and escrow \\
		\end{itemize}
	\subsection{Extensible Authentication Protocol/IEEE 802.1X}
		\begin{itemize}
			\item Authenticate user at network access devices
			\item Extensible authentication protocol
			\item IEEE 802.1X Port Based Network Access Control \\
				-- Supplicant, network access server (NAS), AAA server
		\end{itemize}
	\subsection{Terminal Access Controller Access-Control System}
		\begin{itemize}
			\item TACACS+
			\item Centralizing admin logins
			\item Reliable TCP Transport (over \textbf{port 49})
		\end{itemize}
	\subsection{Token Keys and Static Codes}
		\begin{itemize}
			\item One-time password
			\item Static code -- "dumb" smart cards
			\item Fast Identity Online (FIDO), Universal Second Factor
		\end{itemize}
	\subsection{Open Authentication (OAUTH)}
		\begin{itemize}
			\item HMAC-based one-time password (HOTP)
			\item Time based One-time Passowrd (TOTP)
		\end{itemize}
	\subsection{2-Step Verification}
		\begin{itemize}
			\item Transmit a code via out-of-band channel
			\item Possibility of interception
		\end{itemize}

\section{Biometric Authentication}
	\subsection{Biometric Authentication}
		\begin{itemize}
			\item Enrollment -- sensor and feature extraction
			\item Efficacy rates and considerations \\
				-- \textbf{False rejection rates (FRR)} or Type I error \\
				-- \textbf{False acceptance rates (FAR)} or Type II error \\
				-- \textbf{Crossover Error Rate (CER)} \\
				-- Throughput, failure to enrol, cost/implementation \\
				-- Privacy concerns and accessibility concerns
		\end{itemize}
	\subsection {Fingerprint recognition}
		\begin{itemize}
			\item Fingerprint sensors -- small capacitive cells, vuln to spoofing
			\item Vein Matching(vascular biometrics) -- more complex scanner
		\end{itemize}
	\subsection {Facial Recognition}
		\begin{itemize}
			\item Facial Recognition -- relatively slow, privacy issues, FAR, FRR
			\item Retinal Scan -- pattern of blood vessels, relatively intrusive/complex
			\item Iris scan -- more vulnerable to spoofing
		\end{itemize}
	\subsection {Behavioral Technologies}
		\begin{itemize}
			\item Something you do -- voice recognition, gait, signature
			\item Other uses than authentication -- identification/alerting
		\end{itemize}
	
\section {Implement Identity and account types}
	\subsection {Identity Management Controls}
		\begin{itemize}
			\item Certificates and smart cards
			\item Tokens -- single sign-on, avoids need to authenticate every service
			\item Identity providers
		\end{itemize}
	\subsection {Background Check and On board Policies}
		\begin{itemize}
			\item HR and personnel policies -- recruiting, operation, termination
			\item Background Check
			\item Onboarding -- welcoming, account provisioning, issuing creds, training
			\item Non-Disclosure Agreement (NDA)
		\end{itemize}
	\subsection {Personnel Policies for Privilege Management}
		\begin{itemize}
			\item Mitigate insider threat
			\item Separation of duties -- shared authority
			\item Least Privilege -- assign sufficient permissions only
			\item Job rotation -- distribute institutional knowledge,
				reduce critical dependencies
			\item Mandatory vacations
		\end{itemize}
	\subsection {Offboarding Policies}
		\begin{itemize}
			\item Identity and access management checks
			\item Retrieving company assets
			\item Returning personal assets
			\item Consider shared/generic accounts
		\end{itemize}
	\subsection {Security account types and Credential Management}
		\begin{itemize}
			\item Standard users -- limited privileges, not able to configure
			\item Credential management policies for personnel -- password policies
			\item Guest accounts -- no credentials, must have very limited privileges
		\end{itemize}
	\subsection {Security Group-Based Privileges}
		\begin{itemize}
			\item User-assigned privileges -- unmanageable if large
			\item Group-based privileges -- assign users to relevant groups
		\end{itemize}
	\subsection {Administrator/Root Accounts}
		\begin{itemize}
			\item \textbf{Privileged/admin accounts} -- can change system config
			\item \textbf{Generic/admin/root/superuser} -- often disabled or use restricted
			\item \textbf{Administrator credential policies} -- least amount of privileges
				and use MFA
			\item \textbf{Default Security Groups} -- admin/sudoers file
		\end{itemize}
	\subsection {Service accounts}
		\begin{itemize}
			\item \textbf{Windows Service Accounts} -- system/local/network
			\item \textbf{Linux accounts to run services} -- deny shell access (nologin)
			\item Managing shared service account credentials
		\end{itemize}
	\subsection {Shared/Generic/Device Acccounts and Credentials}
		\begin{itemize}
			\item Shared Accounts -- Accounts whose credentials are shared
			\item Generic Accounts -- created by default, might use default password
			\item Risks from shared and generic accounts -- breaks non-repudiation
			\item Credential policies for devices
			\item Privilege access management software
		\end{itemize}
	\subsection {SSH Keys and Third-party Credentials}
		\begin{itemize}
			\item SSH keys used for remote access -- server holds copy of users pulic keys
			\item Third party credentials -- manage cloud service, highly vulnerable
		\end{itemize}

\section {Account Policies}
	\subsection {Account Attributes and Access Policies}
		\begin{itemize}
		 	\item Account Attributes
		 		\begin{itemize}
					Security ID, account name, credential
					Extended profile attributes
					Per-app settings and files
				\end{itemize}
			\item Access Policies
		\end{itemize}
	\subsection {Account Password Policy Settings}
		\begin{itemize}
			\item Length
			\item Complexity
			\item Aging
			\item History and Use
			\item NIST Guidance
			\item Password Hints
		\end{itemize}
	\subsection {Account Restrictions}
		\begin{itemize}
			\item Network location -- VLAN, IP subnet, remote IP, remote logon
			\item Geolocation -- By IP Address, Location Settings, Geofencing, Geotagging
			\item Time-based restrictions -- Logon hours, Logon duration, Impossible
				travel time/risky login
		\end{itemize}
	\subsection {Account Audits}
		\begin{itemize}
			\item Accounting and account auditing to detect account misuse 
				\begin{itemize}
					\item Use of file permissions to read and modify data
					\item Failed login or resource access attempts
				\end{itemize}
			\item Recertification
				\begin{itemize}
					\item Monitoring use of privileges
					\item Granting/revoking privileges
					\item Communicating between IT/HR
				\end{itemize}
		\end{itemize}
	\subsection {Account Permissions}
		\begin{itemize}
			\item Impact of improperly configured accounts
			\item Escalating and revoking privileges
			\item Permission and auditing tools
		\end{itemize}
	\subsection {Usage Audits}
		\begin{itemize}
			\item Account logon and management events
			\item Process Creation
			\item Object Access (file system/file shares)
			\item Changes to audit policy
			\item Changes to system security and integrity
		\end{itemize}
	\subsection {Account Lockout and Disablement}
		\begin{itemize}
			\item Disablement
				\begin{itemize}
					\item Login disabled until manually reenabled
					\item Combine with remote logoff
				\end{itemize}
			\item Lockout
				\begin{itemize}
					\item Login is prevented for a period then reenabled
					\item Policies to enforce automatic lockout
				\end{itemize}
		\end{itemize}
	\subsection {Discretionary and Role-Based Access Control}
		\begin{itemize}
			\item Access control model -- permissions/rights
			\item Discretionary Access Control
				\begin{itemize}
					\item Based on resource ownership
					\item Access Control Lists(ACLs)
					\item Vulnerable to compromised privileged user accounts
				\end{itemize}
			\item Role-Based Access Control (RBAC)
				\begin{itemize}
					\item Non-discretionary and more centralized control
					\item Based on defining roles then allocating users to roles
					\item Users should only inherit role permissions
				\end{itemize}
		\end{itemize}
	\subsection {File System Security}
		\begin{itemize}
			\item Access Control List (ACL)
			\item Access Control Entry (ACE)
			\item File System Support
			\item Linux permissions and chmod
				\begin{itemize}
					\item Symbolic (rwx)
					\item User, group and world
					\item Octal
				\end{itemize}
		\end{itemize}
	\subsection {Mandatory and Attribute Access Control}
		\begin{itemize}
			\item Mandatory Access Control (MAC)
				\begin{itemize}
					\item Labels and clearance
					\item System policies to restrict access
				\end{itemize}
			\item Attribute-Based Access Control (ABAC)
				\begin{itemize}
					\item Conditional Access
				\end{itemize}
		\end{itemize}
	\subsection {Rule-Based Access Control}
		\begin{itemize}
			\item Non-discretionary
			\item Conditional Access
			\item Privileged access management
		\end{itemize}
	\subsection {Directory Services}
		\begin{itemize}
			\item Database of subjects
			\item Access Control Lists
			\item X.500 and lightweight directory access protocol (LDAP)
				\begin{itemize}
					\item Distinguished names
					\item Attribute=value pairs
				\end{itemize}
		\end{itemize}
	\subsection {Federation and Attestation}
		\begin{itemize}
			\item Federated Identity Management
				\begin{itemize}			
					\item Networks under separate administrative control share users
				\end{itemize}
			\item Identity providers and attestation
			\item Cloud vs On-premises requirements
		\end{itemize}
	\subsection {Security Assertions and Markup Language}
		\begin{itemize}
			\item Open standard for implementing identity and service provider comms
			\item Attestations/assertions
				\begin{itemize}
					\item XML format
					\item signed using xml 
				\end{itemize}
		\end{itemize}
	\subsection {OAuth and OpenID Connect}
		\begin{itemize}
			\item ``User centric'' services better suited for consumer websites
			\item OAuth -- Communicate authorizations rather than explicitly authenticate
		\end{itemize}

\section {Explain the Importance of Personnel Policies}
	\subsection{Conduct Policies}
		\begin{itemize}
			\item Acceptable Use Policy (AUP)
			\item Rules of Behavior and social media analysis
			\item Uses of personally owned devices
				\begin{itemize}
					\item Bring your own device
					\item Shadow IT
				\end{itemize}
			\item Clean desk
		\end{itemize}
	\subsection {User and Role-based Training}
		\begin{itemize}
			\item Impacts and risks from untrained users
			\item Topics for security awareness
			\item Role-based Training
				\begin{itemize}
					\item Appropriate Language
					\item Level of Technical Content
				\end{itemize}
		\end{itemize}
	\subsection {Diversity of Training Techniques}
		\begin{itemize}
			\item Engagement and retention
			\item Training delivery methods
			\item Phishing campaigns
			\item Capture the flag
			\item Computer-based training (CBT)
		\end{itemize}

\section {Implementing Secure Network Designs}
	\subsection {Secure Network Design}
		\begin{itemize}
			\item Problems and weaknesses
				\begin{itemize}
					\item Single point of failure
					\item Complex dependencies
					\item Availability over confidentiality and integrity
					\item Lack of documentation and change control
					\item Overdependence on perimeter security
				\end{itemize}
			\item Best practice design and architecture guides
				\begin{itemize}
					\item Cisco SAFE architecture
					\item Places in the network
				\end{itemize}
		\end{itemize}
	\subsection {Business Workflows and Network Architecture}
		\begin{itemize}
			\item Corporate Network
				\begin{itemize}
					\item Access
					\item Email Server
					\item Mail transfer server
				\end{itemize}
			\item Segmentation
			\item Data flow and access controls
		\end{itemize}
	\subsection {Routing and Switching Protocols}
		\begin{itemize}
			\item Forwarding -- Layer 2, 3
			\item Address Resolution Protocol (ARP) -- Map MAC addresses to IP
			\item Internet Protocol (IP) -- IPv4 and IPv6, network prefix/subnet
				\begin{itemize}
					\item IPv4 with 192.168... is private
					\item IPv6 fe80:: is private
				\end{itemize}
			\item Routing protocols -- communicate routing table updates
		\end{itemize}
	\subsection {Network Segmentation}
		\begin{itemize}
			\item Network Segmentation -- nodes communicate at layer 2
			\item Implement network segments -- unmanaged switches, VLANs for managed
			\item Layer 3 subnets -- Map subnets to VLANs
		\end{itemize}
	\subsection {Network Topology and Zones}
		\begin{itemize}
			\item Physical and network topologies
			\item Zones represent isolated segments
			\item Traffic between zones is subject to filtering by a firewall
			\item Main zone types -- intranet(private), extranet, internet(public)
			\item Enterprise architecture zones
		\end{itemize}
	\subsection {Demilitarized Zones}
		\begin{itemize}
			\item DMZs isolate hosts that are Internet-facing
			\item Communications through the DMZ should not be allowed
			\item Ideally use proxies to rebuild packets for forwarding
			\item Bastion Hosts
				\begin{itemize}
					\item Not fully trusted by internal network
					\item Run minimal services
					\item Do not store local network account credentials
				\end{itemize}
		\end{itemize}
	\subsection {Screened Host}
		\begin{itemize}
			\item Screened host -- local network screened by a firewall
		\end{itemize}
	\subsection {Implications of IPv6}
		\begin{itemize}
			\item Enabled by default configuration issues
			\item Map IPv6 address space to appropriate security zones
			\item Configure secure IPv6 firewall rules
			\item Typically no need for address translation
		\end{itemize}
	\subsection {Other Secure Network Design Considerations}
		\begin{itemize}
			\item Data center and cloud design requirements
			\item East-west traffic -- within data center
			\item North-south traffic -- leaving and entering data center
			\item Zero trust -- do not rely solely on perimeter security
				\begin{itemize}
					\item Continuous/context-based auth
					\item Microsegmentation
				\end{itemize}
		\end{itemize}

\section {Implement Secure Switching and Routing}
	\subsection {MITM and Layer 2 Attacks}
		\begin{itemize}
			\item MITM -- intercept and modify communications
			\item Layer 2 Attacks -- easy to change MAC value
		\end{itemize}
	\subsection {Loop Prevention}
		\begin{itemize}
			\item Spanning Tree Protocol (STP)
			\item Broadcast Storm Prevention
			\item Bridge Protocol Data Unit (BPDU) Guard -- disable port if STP is detected
		\end{itemize}
	\subsection {Physical Port Security and MAC Filtering}
		\begin{itemize}
			\item Physical Port Security
				\begin{itemize}
					\item secure switch hardware
					\item physically disconnect unused ports
				\end{itemize}
			\item MAC address limiting and filtering
		\end{itemize}

\section {Implement Secure Wireless Infrastructure}
	\subsection {Wireless Network Installation Considerations}
		\begin{itemize}
			\item Ensure max availability
			\item Wireless access point (WAP) placement
			\item Site surveys and heat maps
		\end{itemize}
	\subsection {Controller and Access Point Security}
		\begin{itemize}
			\item Hardware and Software
			\item Fat vs Thin WAPs
		\end{itemize}
	\subsection {Extensible Authentication Protocol}
		\begin{itemize}
			\item Designed for interoperable security devices
		\end{itemize}

\section {Implement Load Balancers}
	\subsection {Load Balancing}
		\begin{itemize}
			\item Distributes requests across farm or pool of servers
				\begin{itemize}
					\item Layer 4 -- TCP, IP
					\item Layer 7 -- Application level (content switch)
				\end{itemize}
			\item Scheduling
				\begin{itemize}
					\item Round robin
					\item Fewest existing connections
					\item Weighting
					\item Hearbeat and health checks
				\end{itemize}
			\item Source IP affinity
				\begin{itemize}
					\item Persistence -- works by setting a cookie
				\end{itemize}
		\end{itemize}

\section {Implement Firewalls and Proxy Servers}
	\subsection {Packet Filtering Firewalls}
		\begin{itemize}
			\item Enforce a network to use Access Control Lists (ACLs)
			\item Act to deny (block or drop), log or accept a packet
			\item Inspect headers
				\begin{itemize}
					\item Source and destination IP address
					\item Inbound, outbound, or both
					\item Source and destinations port
				\end{itemize}
			\item Inbound, outbound, or both
			\item Stateless
		\end{itemize}
	\subsection {Stateful inspection firewalls}
		\begin{itemize}
			\item Stores connection information
			\item Layer 4
				\begin{itemize}
					\item TCP Handshake
					\item New vs Established and related connection
				\end{itemize}
			\item Application Layer (Layer 7)
				\begin{itemize}
					\item Validate protocol
					\item Match threat signature
				\end{itemize}
		\end{itemize}
	\subsection {Firewall Implementation}
		\begin{itemize}
			\item Firewall Appliances
				\begin{itemize}
					\item Routed (Layer 3)
					\item Bridged/transparent (Layer 2)
					\item Router/Firewall
				\end{itemize}
			\item Application Firewalls
				\begin{itemize}
					\item Host-based (Personal)
					\item Application firewall
					\item Network operation (NOS) firewall
				\end{itemize}
		\end{itemize}
	\subsection {Proxies and Gateways}
		\begin{itemize}
			\item Forward proxy server
				\begin{itemize}
					\item Opens connections with external on behalf of internal clients
					\item Application-specific filters
				\end{itemize}
			\item Reverse proxy server
				\begin{itemize}
					\item Proxy opens connections with internal servers on behalf of
						external clients
				\end{itemize}
		\end{itemize}
	\subsection {Access control lists}
		\begin{itemize}
			\item Least access
			\item Top to bottom processing
			\item Implicit Deny
			\item Explicit Deny all
			\item Criteria for rules (tuples)
			\item Documenting and testing configuration
		\end{itemize}
	\subsection {Network Address Translation}
		\begin{itemize}
			\item Translate private IP address to public IP address
			\item Source NAT
				\begin{itemize}
					\item Static and dynamic NAT
					\item Overloaded NAT/Network Address Port Translation (NAPT)
				\end{itemize}
			\item Destination NAT/port forwarding
				\begin{itemize}
					\item Advertise a resource using a global IP address but forward
						it to a local IP address
				\end{itemize}
		\end{itemize}
	\subsection {Virtual Firewalls}
		\begin{itemize}
			\item Hypervisor-based -- built-in filtering
			\item Virtual appliance -- deployed as a virtual machine
			\item Multiple context -- firewall appliance running multiple instances
			\item East-west security design and microsegmentation
		\end{itemize}
	\subsection {Open-source vs Proprietary}
		\begin{itemize}
			\item Source code inspection and supply chain issues
			\item Support arrangement and subscription features
		\end{itemize}

\section {Implement Network Security Monitoring}
	\subsection {Network-based Intrusion Detection Systems}
		\begin{itemize}
			\item Intrusion Detection Systems
			\item Network Sensor captures traffic
			\item Detection engine performs real-time analysis of indicators
			\item Passive logging/alerting
		\end{itemize}
	\subsection {TAPs and Port Mirrors}
		\begin{itemize}
			\item Sensor placement
			\item Switched port analyzer (SPAN)/mirror port
			\item Passive Test Access Point
			\item Active TAP
			\item Aggregation TAP
		\end{itemize}
	\subsection {Network-based Intrusion Prevention Systems}
		\begin{itemize}
			\item Intrusion Prevention System (IPS)
			\item Active response to threats
				\begin{itemize}
					\item Reset Session
					\item Apply firewall filters
					\item Bandwidth throttling
					\item Packet modification
					\item Run a script or other process
				\end{itemize}
			\item Anti-virus scanning/content filtering
			\item Inline placement--risk of failure
		\end{itemize}
	\subsection {User based detection}
		\begin{itemize}
			\item Analysis Engine
			\item Signature-based detection
				\begin{itemize}
					\item Pattern matching
					\item Database of known attack signatures
					\item Must be updated with latest definition
					\item Many attack tools do not conform to specific signatures
				\end{itemize}
		\end{itemize}
	\subsection {Behavior and Anomaly-based Detection}
		\begin{itemize}
			\item Behavioral-based detection
				\begin{itemize}
					\item Train sensor with baseline normal behavior
					\item Network behavior and anomaly detection (NBAD)
					\item Heuristics (learning from experience)
					\item Statistical model of behavior
					\item Machine learning assisted analysis
				\end{itemize}
			\item Anomaly-based detection as irregularity in packet construction
		\end{itemize}
	\subsection {Next Generation Firewalls and Content Filters}
		\begin{itemize}
			\item Next-Generation firewall -- application-aware filtering, user
				account-based filtering, IPS, cloud inspection
			\item Unified Threat Management (UTM)
			\item Content/URL Filter
				\begin{itemize}
					\item Focuses on outgoing user traffic
					\item Content block lists and allow lists
					\item Time-based restrictions
					\item Secure web gateway(SWG)
				\end{itemize}
		\end{itemize}
	\subsection {Host-based Intrusion Detection Systems}
		\begin{itemize}
			\item Host-based IDS -- Network, log, and file system monitoring for endpoints
			\item File Integrity Monitoring (FIM)
				\begin{itemize}
					\item Cryptographic hash or file signature verifies integrity of files
					\item Compare hashes manually
					\item Windows file Protection/sfc
					\item Tripwire and OSSEC
				\end{itemize}
		\end{itemize}
	\subsection {Web Application Firewalls}
		\begin{itemize}
			\item Able to inspect HTTP Traffic
			\item Matches suspicious code to vulnerability database
			\item Can be implemented as software on host or as appliance
		\end{itemize}
	\subsection {Security Information and Event Management}
		\begin{itemize}
			\item Log collection
				\begin{itemize}
					\item Agent-based -- Local agent to forward logs
					\item Listener/collector -- protocol based remote log
					\item Sensor -- Packet capture and traffic flow data
				\end{itemize}
			\item Log aggregation
				\begin{itemize}
					\item Consolidation of multiple log formats to facilitate search/query
					\item Normalization of fields
					\item Time synchronization
				\end{itemize}
		\end{itemize}
	\subsection {Analysis and Report Review}
		\begin{itemize}
			\item Correlation
				\begin{itemize}
					\item Relating security data and threat intelligence
					\item Alerting of indicators of compromise
					\item Basic rules vs machine learning
				\end{itemize}
			\item User and entity behavior analysis (UEBA)
			\item Sentiment analysis
			\item Security orchestration, automation and response(SOAR)
		\end{itemize}
	\subsection {File Manipulation}
		\begin{itemize}
			\item cat -- view contents of one or more files
			\item head or tail -- view first and last lines of file
			\item logger -- write system input to system log
		\end{itemize}
	\subsection {Regular Expressions and grep}
		\begin{itemize}
			\item Regular expression syntax -- Search operators, quantifiers
			\item grep -- Searches file contents
		\end{itemize}

\section {Implementing Secure Network Protocols}
	\subsection {Network Address Allocation}
		\begin{itemize}
			\item Dynamic vs Static IP address management
			\item Dynamic Host Configuration Protocol (DHCP)
			\item Prevent rogue DHCP Servers
			\item Prevent DoS attacks (starvation) by rogue clients
			\item Secure administration interface
		\end{itemize}
	\subsection {Domain Name Resolution}
		\begin{itemize}
			\item System for resolving host names and domain labels to IP addresses
			\item Domain hijacking -- gain control of domain registration
			\item Uniform Resource Locator (URL) redirection -- abuse of HTTP requests
			\item Domain reputation -- monitor blocklists/reputation lists for abuse
		\end{itemize}
	\subsection {DNS Poisoning}
		\begin{itemize}
			\item Man In the Middle -- rogue DNS server intercepts queries
			\item DNS client cache poisoning -- HOSTS file
			\item DNS server cache poisoning
		\end{itemize}
	\subsection {DNS Security}
		\begin{itemize}
			\item DNS server security
			\item DNS Server Security Extensions (DNSSEC)
		\end{itemize}
	\subsection {Secure Directory Services}
		\begin{itemize}
			\item Directory Services and Lightweight Directory Access Protocol (LDAP) --
				port 389
			\item Binding Methods
				\begin{itemize}
					\item None
					\item Simple Authentication
					\item Simple Authentication and Security Layer (SASL)
					\item LDAPS (TLS over port 636)
				\end{itemize}
			\item Access control policy
				\begin{itemize}
					\item Read-only
					\item Read/Write
				\end{itemize}
		\end{itemize}
	\subsection {Time Synchronization}
		\begin{itemize}
			\item Time critical services
				\begin{itemize}
					\item Authentication
					\item Logging
					\item Task scheduling/backup
				\end{itemize}
			\item Network time protocol (NTP)
				\begin{itemize}
					\item Stratum 1 Servers
					\item Stratum 2 Servers
					\item Simple NTP (Clients)
				\end{itemize}
		\end{itemize}
	\subsection {Simple Network Management Protocol Security}
		\begin{itemize}
			\item Simple Network Management Protocol (SNMP)
			\item SNMP v1 and v2 feature no or weak authentication and no privacy
			\item SNMP v3 encryption and authentication
		\end{itemize}

\section {Implement Secure Application Protocols}
	\subsection {HTTP and Web Services}
		\begin{itemize}
			\item HTTP Headers and Payload
			\item Web services/applications
				\begin{itemize}
					\item Forms mechanism allows client to upload data to server
					\item Statelesss protocol but expanded with cookies and scripting
				\end{itemize}
		\end{itemize}
	\subsection {Transport Layer Security}
		\begin{itemize}
			\item SSL/TLS -- Communications secured using host certificates
			\item SSL/TLS versions
			\item Cipher Suites
				\begin{itemize}
					\item Key exchange -- HMAC \texttt{ECDHE-RSA-AES128-GCM-SHA256}
					\item TLS 1.3 uses shortened suites
				\end{itemize}
		\end{itemize}
	\subsection {API Considerations}
		\begin{itemize}
			\item Application Programming Interface
			\item API Keys
				\begin{itemize}
					\item Static keys
					\item Authorization and Authentication via SAML/OAuth
				\end{itemize}
		\end{itemize}
	\subsection {Subscription Services}
		\begin{itemize}
			\item News and subscription services
			\item Provide secure access
			\item News feed security
		\end{itemize}
	\subsection {File transfer services}
		\begin{itemize}
			\item SSH FTP (SFTP) -- run FTP over SSH on port 22
			\item FTP over SSL (FTPS)
		\end{itemize}
	\subsection {Email Services}
		\begin{itemize}
			\item Simple Mail Transfer Protocol (SMTP)
			\item Mailbox access protocols
				\begin{itemize}
					\item Post Office Protocol (POP3)
					\item Internet Message Access Protocols (IMAP)
					\item Secure ports
						\begin{itemize}
							\item POP3S port 995
							\item IMAP port 993
						\end{itemize}
				\end{itemize}
		\end{itemize}
	\subsection {Secure/Multipurpose Internet Mail Extensions (S/MIME)}
		\begin{itemize}
			\item End-to-end encryption for message contents
			\item Authentication and confidentiality using PKI certificates
			\item Correspondents must exchange and trust certificates
		\end{itemize}
	\subsection {Voice and Video Protocol Security}
		\begin{itemize}
			\item VOIP, web conferencing, and video teleconferencing (VTC)
				\begin{itemize}
					\item Session control
					\item Data transport
					\item Quality of service
				\end{itemize}
			\item Session Initiation Protocol (SIP)
			\item Secure Real-time Transport Protocol (SRTP) -- call data confidentiality
		\end{itemize}

\section {Implement Secure Remote Access Protocols}
	\subsection {Remote Access Architecture}
		\begin{itemize}
			\item Remote (Client) Access VPN
			\item Site-to-Site VPN
		\end{itemize}
	\subsection {Transport Layer Security VPN}
		\begin{itemize}
			\item Use TLS to negotiate a secure communication, auth'ed by PKI Certs
			\item Tunnel network traffic over TLS
			\item Can use TCP or UDP
			\item OpenVPN
				\begin{itemize}
					\item TAP/bridged mode
					\item TUN/routed mode
				\end{itemize}
			\item Secure Sockets Tunneling
		\end{itemize}
	\subsection {Internet Protocol Security (IPSec)}
		\begin{itemize}
			\item Network Layer Security
			\item Provides confidentiality and/or Integrity
			\item Authentication Header (AH)
				\begin{itemize}
					\item Signs packet but does not encrypt payload
					\item Provides authentication/integrity only
				\end{itemize}
			\item Encapsulation Security Payload (ESP)
				\begin{itemize}
					\item Provides confientiality and/or integrity
				\end{itemize}
		\end{itemize}
	\subsection{IPSec Transport and Tunnel Modes}
		\begin{itemize}
			\item Transport Mode -- host-to-host connections on a private network
			\item Tunnel Mode -- between gateways
		\end{itemize}
	\subsection {Internet Key Exchange}
		\begin{itemize}
			\item IKE
			\item Security Association (SA)
			\item Endpoints must communicate a shared secret and confirm identity
		\end{itemize}
	\subsection {VPN Client Configuration}
		\begin{itemize}
			\item Native VPN client or third-party software install
			\item Configuration
				\begin{itemize}
					\item VPN gateway address
					\item Security type and user credentials
					\item Client certificate install
				\end{itemize}
			\item Always-on VPN
			\item Split tunnel
			\item Full tunnel -- internet access is mediated by the corporate network
		\end{itemize}
	\subsection {Remote Desktop}
		\begin{itemize}
			\item GUI-based remote terminal software
			\item Remote Desktop Protocol (RDP)
			\item HTML5/Clientless
		\end{itemize}
	\subsection {Out-of-Band Management and Jump Servers}
		\begin{itemize}
			\item Secure admin workstations (SAWs)
			\item OOB Management
				\begin{itemize}
					\item Serial/modem/console port
					\item Virtual terminal
					\item Separate cabling or VLAN isolation
				\end{itemize}
			\item Jump servers
				\begin{itemize}
					\item Single host accepts SSH or RDP connections from SAWs
					\item Forwards connection to app servers
					\item App servers only accept connections from jump servers
				\end{itemize}
		\end{itemize}
	\subsection {Secure Shell (SSH)}
		\begin{itemize}
			\item Remote administration with public key cryptography security
			\item Host key identifies server
			\item Client authentication
				\begin{itemize}
					\item Username/Password
					\item Public Key Authentication
					\item Kerberos
				\end{itemize}
			\item Key Management
			\item SSH Commands
		\end{itemize}

\section {Implementing Host Security Solutions}
	\subsection {Implement Secure Firmware}
		\subsubsection {Hardware Root of Trust}
			\begin{itemize}
				\item Hardware root of trust/anchor
				\item Attestation
				\item Trusted Platform Module (TPM)
					\begin{itemize}
						\item Hardware-based Storage of Cryptographic Data
						\item Endorsement Key
						\item Subkeys used in key storage, signature and
							encryption operations
						\item Ownership secured via password
					\end{itemize}
			\end{itemize}
		\subsubsection {Boot Integrity}
			\begin{itemize}
				\item Unified Extensible Firmware Interface (UEFI)
				\item Secure Boot -- validate digital signatures before running boot
					loader or kernel
				\item Measured Boot -- use TPM to measure hashes
				\item Attestation -- Report boot metrics and signature
			\end{itemize}
		\subsubsection {Drive Encryption}
			\begin{itemize}
				\item Full Disk Encryption (FDE)
					\begin{itemize}
						\item Encryption key secured with user password
						\item Secure Storage for key in TPM or USB
					\end{itemize}
				\item Self-encrypting Drives (SED)
					\begin{itemize}
						\item Data/media encrpytion key
						\item Authentication Key(AK) or key encrypting key (KEK)
						\item Opal specification compliant
					\end{itemize}
			\end{itemize}
		\subsubsection {USB and Flash Drive Security}
			\begin{itemize}
				\item BadUSB -- Exposes potential of malicious firmware
				\item Sheep dip -- Sandbox system for testing new/suspect devices
			\end{itemize}
		\subsubsection {Third Party Risk Management}
			\begin{itemize}
				\item Supply chain and vendors
					\begin{itemize}
						\item End to end process supplying, manufacturing, distributing
							and finally releasing goods and services to a customer
						\item Consider implications of using second-hand equipment
					\end{itemize}
				\item Vendor vs business partners
			\end{itemize}
		\subsubsection {End of Life Systems and Lack of Vendor Support}
			\begin{itemize}
				\item Support lifecycles
				\item End of Life (EOL)
					\begin{itemize}
						\item Product is no longer sold to new customers
						\item Availability of spares and updates is reduced
					\end{itemize}
				\item End of Service Life (EOSL)
					\begin{itemize}
						\item Product is no longer supported
					\end{itemize}
				\item Lack of vendor support
					\begin{itemize}
						\item Abandonware
						\item Software and peripherals/devices
					\end{itemize}
			\end{itemize}
		\subsubsection {Organizational Security Agreements}
			\begin{itemize}
				\item Memorandum of Understanding (MOU) -- intent of working together
				\item Business Partnership Agreements (BPA) -- establish relationship
				\item Non-disclosure Agreement (NDA) -- govern use and storage of
					confidential and private information
				\item Service Level Agreement (SLA) -- metrics for service delivery and
					performance (negotiations of uptime/downtime)
				\item Measurement analysis (MSA) -- data collection and statistical
					methods used for quality management
			\end{itemize}
	\subsection {Implement Endpoint Security}
		\subsubsection {Host Hardening}
			\begin{itemize}
				\item Reducing attack surface
				\item Interface -- network and peripheral connections and hardware ports
				\item Services -- Software that allows client connections
				\item Application service ports
					\begin{itemize}
						\item TCP and UDP ports
						\item Disable application service or use firewall to control
							access
						\item Detect non-standard usage
					\end{itemize}
				\item Encryption for persistent storage
			\end{itemize}
		\subsubsection {Baseline Configuration and Registry Settings}
			\begin{itemize}
				\item OS/host role -- network appliance, server, client
				\item Configuration baseline template
				\item Registry settings and group policy objects (GPOs)
				\item Malicious registry changes
				\item Baseline deviation reporting
			\end{itemize}
		\subsubsection {Patch Management}
			\begin{itemize}
				\item All types of OS, application, and firmware code potentially
					contains vulnerabilities
				\item Patch management essential for mitigating these vulnerabilities
				\item Update policies and schedule
					\begin{itemize}
						\item Apply all latest -- autoschedule
						\item Only apply specific patches
						\item Third-party patches
					\end{itemize}
				\item Scheduling updates
				\item Managing unpatchable systems
			\end{itemize}
		\subsubsection {Endpoint Management}
			\begin{itemize}
				\item AV/Antimalware
				\item Host-based Intrusion Detection/Prevention System (HIDS/HIPS)
				\item Endpoint Protection Platform (EPP)
				\item Data Loss Protection (DLP) -- block copy or transfer of 
					confidential data
				\item Endpoint protection deployment
			\end{itemize}
		\subsubsection {Next-Generation Endpoint Protection}
			\begin{itemize}
				\item Endpoint detection and Response (EDR)
				\item Next-generation firewall integration
			\end{itemize}
		\subsubsection {Antivirus Response}
			\begin{itemize}
				\item Signature-based detection and heuristics
				\item Common malware enumeration and classification
				\item Manual remediation advice
				\item Advanced malware tools
				\item Sandboxing
			\end{itemize}
	\subsection {Embedded Systems}
		\subsubsection {Embedded Systems}
			\begin{itemize}
				\item Computer systems with dedicated function
				\item Static Environment
				\item Cost, power, and compute constraints
				\item Crypto, authentication and implied trust constraints
				\item Network and range constraints
			\end{itemize}
		\subsubsection {Logic Controllers for Embedded Systems}
			\begin{itemize}
				\item Programmable Logic Controllers(PLC)
				\item System on a Chip (SoC)
					\begin{itemize}
						\item Processors, controllers, and devices all provided
							on a single package
						\item Raspberry Pi, Arduino
					\end{itemize}
				\item Field Programmable Gate Array (FPGA)
				\item Real-time Operating System (RTOS)
					\begin{itemize}
						\item Designed to be ultra-stable
						\item Real time scheduling
					\end{itemize}
			\end{itemize}
		\subsubsection {Embedded Systems Communications Considerations}
			\begin{itemize}
				\item Operational Technology (OT) networks
				\item Cellular networks/baseband radio
					\begin{itemize}
						\item Narrowband IOT (NB-IOT)
						\item LTE Machine Type Communication
						\item Subscriber Identity Module (SIM) Cards
						\item Encryption and backhaul
					\end{itemize}
				\item Z-wave and Zigbee
			\end{itemize}
		\subsubsection {Industrial Control Systems}
			\begin{itemize}
				\item Availability, integrity, confidentiality (AIC triad)
					-- Availability comes first in industrial control systems
				\item Workflow and process automation
					\begin{itemize}
						\item Industrial control systems (ICS)
						\item Plant devices and embedded PLCs
						\item OT network
						\item Electromechanical components and sensors
						\item Human machine interface (HMI)
						\item Data historian
					\end{itemize}
				\item Supervisory Control and Data Acquisition (SCADA)
					\begin{itemize}
						\item Runs on PCs to gather data and perform monitoring
						\item Manage large-scale, multiple site communications
					\end{itemize}
			\end{itemize}
		\subsubsection {Internet of Things}
			\begin{itemize}
				\item Machine to Machine communication
				\item Hub/control system
				\item Smart devices
				\item Wearables
				\item Sensors
				\item Vendor security management
			\end{itemize}
		\subsubsection {Specialized Systems for Facility Automation}
			\begin{itemize}
				\item Building automation system (BAS)
					\begin{itemize}
						\item Smart Buildings
						\item Process and memory vulnerabilities
						\item Credentials embedded in application code
						\item Code injection 
					\end{itemize}
				\item Smart meters
				\item Surveillance systems
					\begin{itemize}
						\item Physical access control system
						\item Risks from third-party provision
						\item Abuse of cameras
					\end{itemize}
			\end{itemize}
		\subsubsection {Specialized Systems in IT}
			\begin{itemize}
				\item Multifunction Printer (MFP)
				\item Voice Over IP (VOIP)
				\item Shodan
			\end{itemize}
		\subsubsection {Specialized systems for Vehicles and Drones}
			\begin{itemize}
				\item UAV/Drones
				\item Computer controlled or assisted engine, steering, brakes
				\item In-vehicle entertainment and navigation
				\item Controller area network (CAN) serial communications buses
					\begin{itemize}
						\item Onboard Diagnostics (OBD-II) Module
						\item Access via cellular or Wifi
					\end{itemize}
			\end{itemize}
		\subsubsection {Specialized Systems for Medical Devices}
			\begin{itemize}
				\item Used in hospitals and clinics but also at home by patients
				\item Potentially unsecure protocols and control systems
				\item Use compromised devices to point to networks -- stealing PHI
				\item Ransom by threatening to disrupt services
				\item Kill or injure patients
			\end{itemize}
		\subsubsection {Security for Embedded Systems}
			\begin{itemize}
				\item Network Segmentation
					\begin{itemize}
						\item Strictly restrict access to OT networks
						\item Incresed monitoring for SCADA hosts
					\end{itemize}
				\item Wrappers -- use IPSec for authentication and integrity and
					confidentiality
				\item Firmware code control -- supply chain risks
				\item Inability to patch
			\end{itemize}

\section {Implementing Secure Mobile Solutions}
	\subsection {Implement Mobile Device Management}
		\subsubsection {Mobile Device Deployment Models}
			\begin{itemize}
				\item Bring your own device (BYOD)
				\item Corporate owned, business owned (COBO)
				\item Corporate owned, personally-enabled (COPE)
				\item Choose your own device (CYOD)
				\item Virtual desktop infrastructure (VDI)
			\end{itemize}
		\subsubsection {Enterprise Mobility Management}
			\begin{itemize}
				\item Apply security policies to the use of mobile devicess in the
					enterprise
				\item Visibility over use and configuration
				\item Enterprise mobility management (EMM)
				\item Mobile device management (MDM) -- network enrollment,
					device functions
				\item Mobile application management (MAM) 
			\end{itemize}
		\subsubsection {iOS in the Enterprise}
			\begin{itemize}
				\item App development
					\begin{itemize}
						\item Software Development Kit (macOS only)
						\item App Store
						\item Device Enrollment Program
					\end{itemize}
				\item iOS Vulnerabilities and Patch Management
			\end{itemize}
		\subsubsection {Android in the Enterprise}
			\begin{itemize}
				\item App Store and developer programs
				\item Android vulnerabilies and patch management
				\item Security Enhanced Android (SEAndroid)
			\end{itemize}
		\subsubsection {Mobile Access Control Systems}
			\begin{itemize}
				\item Smartphone authentication
				\item Screen lock
				\item Context-aware authentication
			\end{itemize}
		\subsubsection {Remote Wipe}
			\begin{itemize}
				\item Kill switch
				\item Sets device to factory defaults or clear storage
				\item Initiated from enterprise management software
				\item Thief might be able to keep the device from receiving
					the wipe command
			\end{itemize}
		\subsubsection {Full Device Encryption and External Media}
			\begin{itemize}
				\item iOS device encryption
				\item Android device encryption
				\item External media
				\item MicroSD HSM
			\end{itemize}
		\subsubsection {Location Services}
			\begin{itemize}
				\item Geolocation
				\item Location Services
					\begin{itemize}
						\item Global Positioning System(GPS)
						\item Indoor Positioning Systems(IPS)
					\end{itemize}
				\item Geofencing to apply location-based policies automatically
				\item GPS-tagging
			\end{itemize}
		\subsubsection {Application Management}
			\begin{itemize}
				\item MDM/EMM application use policies
				\item Corporate workspaces
				\item Restricting third-party app stores
				\item Enterprise app development and fulfillment -- sideloading
			\end{itemize}
		\subsubsection {Content Management}
			\begin{itemize}
				\item Privately owned but corporate use issues
				\item Containerization sets up a corporate workplace segmented
				\item Storage segmentation ensures separation of data
				\item Enforcing content management/DLP policies
			\end{itemize}
		\subsubsection {Rooting and Jailbreaking}
			\begin{itemize}
				\item Rooting -- custom firmware/ROM
				\item Jailbreaking -- Principally iOS, tethered jailbreak
				\item Carrier unlocking
				\item Risks to enterprise management
			\end{itemize}
	\subsection {Implement Secure Mobile Device Connections}
		\subsubsection {Cellular and GPS Connection Methods}
			\begin{itemize}
				\item Disable cellular data if unmonitored or unfiltered
				\item Prevent use for data exfiltration
				\item Attacks on cellular connections
				\item Global Positioning Systems (GPS) - GPS/GPS-A
			\end{itemize}
		\subsubsection {Wi-Fi and Tethering Connection Methods}
			\begin{itemize}
				\item Risks from WiFi
					\begin{itemize}
						\item Legacy security methods
						\item Open access points
						\item Rogue access points
					\end{itemize}
				\item Personal Area Network(PAN) technologies
				\item Wi-Fi Direct
				\item Tethering and hotspots
			\end{itemize}
		\subsubsection {Bluetooth Connection Methods}
			\begin{itemize}
				\item Device discovery
				\item Authentication and authorization -- pairing mechanism
				\item Malware and exploits
					\begin{itemize}
						\item Bluebourne
						\item Bluejacking -- sending unsolicited text messages
						\item Bluesnarfing -- exploit to steal info from phones
						\item Rogue firmware peripheral devices
					\end{itemize}
			\end{itemize}
		\subsubsection {Infrared and RFID Connection Methods}
			\begin{itemize}
				\item Infrared -- IR blaster/sensor
				\item Radio Frequency ID (RFID)
					\begin{itemize}
						\item Usually unpowered tags
						\item Transmit when in range of reader
						\item Skimming attack
						\item Encrypt sensitive information
					\end{itemize}
			\end{itemize}
		\subsubsection {Near Field Communications and Mobile Payment Services}
			\begin{itemize}
				\item NFC
				\item Connection configuration/bump
				\item Mobile wallet apps
				\item Eavesdropping/skimming
				\item Denial of service
			\end{itemize}
		\subsubsection {USB Connection Methods}
			\begin{itemize}
				\item USB OTG allows a port to function as a device or hub
				\item USB with malicious firmware might be able to perform an exploit
				\item Juice Jacking
			\end{itemize}
		\subsubsection {SMS/MMS/RCS and Push Notifications}
			\begin{itemize}
				\item Short message service (SMS) -- exploits against SMS
				\item Multimedia message service (MMS)
				\item Rich communication services (RCS) -- WhatsApp, Signal
				\item Push Notifications -- apps diplay alerts on mobile phones
			\end{itemize}
		\subsubsection {Firmware Over-the-Air Updates}
			\begin{itemize}
				\item Baseband updates and radio firmware
				\item Over the Air (OTA) update delivery
				\item Risks from rooted/jailbroken devices
				\item Risks from highly targeted attacks
			\end{itemize}
		\subsubsection {Microwave Radio Connection Methods}
			\begin{itemize}
				\item Backhaul link from cell tower to provider network
				\item Private links between premises
				\item Point-to-point microwave
				\item Point-to-multipoint microwave
				\item Other types of multipoint
			\end{itemize}

\section {Implement Secure Application Attacks}
	\subsection {Analyze Indicators of Application Attacks}
		\subsubsection {Application Attacks}
			\begin{itemize}
				\item Attacks that target vulnerabilities in application code or
					architecture/design
				\item Privilege escalation
					\begin{itemize}
						\item Get privileges from target vulnerable process to
							run arbitrary code
						\item Remote execution
						\item Vertical and horizontal privilege escalation
					\end{itemize}
				\item Error handling
				\item Improper input handling
			\end{itemize}
		\subsubsection {Overflow Vulnerabilities}
			\begin{itemize}
				\item Buffer overflow
					\begin{itemize}
						\item Buffer is memory allocated to application
						\item Overflows can allow arbitrary code to execute
					\end{itemize}
				\item Integer Overflow
					\begin{itemize}
						\item Cause application to calculate values that are
							out-of-bounds
						\item Could use to cause crash or use in buffer overflow attack
					\end{itemize}
			\end{itemize}
		\subsubsection {Null Pointer Dereferencing and Race Conditions}
			\begin{itemize}
				\item Pointers are used in C/C++ to refer to memory locations
				\item Dereferencing occurs when the program tries to read or write the
					location to the pointer
				\item If the location is null or invalid, the process will crash
				\item Race condition
					\begin{itemize}
						\item Execution depends on timing and sequence of events
					\end{itemize}
				\item Time of check/time of use(TOCTOU)
					\begin{itemize}
						\item Environment is manipulated to change a resource after
							checking but before use
					\end{itemize}
			\end{itemize}
		\subsubsection {Memory Leaks and Resource Exhaustion}
			\begin{itemize}
				\item Memory Leaks
					\begin{itemize}
						\item Process allocates memory locations, but never releases them
						\item Can cause host to run out of memory
						\item Could be faulty code or could be malicious
					\end{itemize}
				\item Resource Exhaustion
					\begin{itemize}
						\item CPU Time, system memory allocation, fixed disk capacity,
							and network utilization
						\item Spawning activity to use up these resources
					\end{itemize}
			\end{itemize}
		\subsubsection {DLL Injections and Driver Manipulation}
			\begin{itemize}
				\item Dynamic Link Library(DLL) implements some function that multiple
					functions can use
				\item DLL injection forces a process to load a malicious DL
				\item Refactoring might allow code obfuscation to elude antivirus
				\item Shim -- exploit application compatibility framework to allow malware
					to persist on host
			\end{itemize}
		\subsubsection {Pass the Hash Attack}
			\begin{itemize}
				\item Exploiting cached credentials
				\item Windows host cache credentials in memory as NTLM hashes
				\item Local malicious process with administrator privileges can dump
					themselves in
				\item Detection through security log events
			\end{itemize}
		\subsubsection {Uniform Resource Locator Analysis}
			\begin{itemize}
				\item URL format
				\item HTTP Methods
					\begin{itemize}
						\item TCP Connections
						\item GET, POST, PUT, HEAD
						\item POST or PUT
						\item URL query parameters
						\item Fragment/anchor ID
						\item HTTP response codes
					\end{itemize}
				\item Percent encoding
			\end{itemize}
		\subsubsection {Application Programming Interface Attacks}
			\begin{itemize}
				\item API calls and parameters
				\item Must only be with HTTPS encryption
				\item Common weakness and vulnerabilities
					\begin{itemize}
						\item Inefficient secrets management
						\item Lack of input validation
						\item Error messages leaking information
						\item Denial of Service
					\end{itemize}
			\end{itemize}
		\subsubsection {Replay Attacks}
			\begin{itemize}
				\item Resubmitting or guessing authorization tokens
				\item Session management cookies
				\item Replay cookie to obtain authentication session
				\item Secure cookies
			\end{itemize}
		\subsubsection {Session hijacking and CSRF}
			\begin{itemize}
				\item Cookie hijacking and session prediction
				\item Client-side/cross-site (CSRF/XSRF) request forgery
					\begin{itemize}
						\item Passes URL to another site where the user has an
							authenticated session
						\item Confused deputy
					\end{itemize}
				\item Clickjacking -- add invisible layer to intercept/redirect
					click events
				\item SSL Strip
					\begin{itemize}
						\item Exploit redirect from HTTP to HTTPS
						\item Sites should no longer be using HTTP
						\item HSTS -- HTTP Script Transport Security
					\end{itemize}
			\end{itemize}
		\subsubsection {Cross-Site Scripting (XSS)}
			\begin{itemize}
				\item Attacker injects code in trusted site that will be executed
					in client browser
				\item Non-persistent/reflected
				\item Persistent/stored XSS
				\item Client-side scripts
			\end{itemize}
		\subsubsection {Structured Query Language Injection Attacks}
			\begin{itemize}
				\item Client-side vs server-side attacks
				\item Injection-type attacks
				\item SQL statements
				\item SQL injection
			\end{itemize}
		\subsubsection {XML and LDAP Injection Attacks}
			\begin{itemize}
				\item XML injection
					\begin{itemize}
						\item XML Tagged documents
						\item XML External Entity (XXE)
					\end{itemize}
				\item LDAP Injection
			\end{itemize}
		\subsubsection {Directory Traversal and Command Injection Attacks}
			\begin{itemize}
				\item Directory traversal -- obtain access to files outside of root dir
				\item Command injection -- cause server to run OS commands
			\end{itemize}
		\subsubsection {Server side request forgery}
			\begin{itemize}
				\item Cause a server to make API calls to HTTP requests with arbitrary
					parameters
				\item Variety of exploit techniques and aims
			\end{itemize}
	\subsection {Secure Coding Practices}
		\subsubsection {Secure Coding Techniques}
			\begin{itemize}
				\item Security development lifecycles and best practice guides
				\item Input Validation
					\begin{itemize}
						\item User-generated data form controls
						\item Passed by another program
						\item Document and test all types of user/API input
					\end{itemize}
				\item Normalization and output coding
			\end{itemize}
		\subsubsection {Server-side vs Client-side Validation}
			\begin{itemize}
				\item Client-side execution
					\begin{itemize}
						\item Code is run by the browser
						\item Document Object Model (DOM) scripting
						\item Might send a request to the server, but the request
							is constructed by the client
					\end{itemize}
				\item Server-side execution
					\begin{itemize}
						\item Code is run by the server
					\end{itemize}
				\item Client side input validation
			\end{itemize}
		\subsubsection {Web Application Security}
			\begin{itemize}
				\item Secure cookies
					\begin{itemize}
						\item Avoid using persistent cookies for session authentication
						\item Set the Secure attribute
						\item Set the HTTPOnly attribute
						\item Use the SameSite attribute
					\end{itemize}
				\item Response Headers
					\begin{itemize}
						\item HTTP Strict Transport Security
						\item Content Security Policy (CSP)
						\item Cache-control
					\end{itemize}
			\end{itemize}
		\subsubsection {Data Exposure and Memory Management}
			\begin{itemize}
				\item Data Exposure
					\begin{itemize}
						\item Allowing privileged data to be read without authorization
						\item Lack of encryption
					\end{itemize}
				\item Error Handling
					\begin{itemize}
						\item Struction Exception Handler (SEH)
						\item Prevent use of error conditions for arbitrary code/injection
					\end{itemize}
				\item Memory Management
					\begin{itemize}
						\item Use of unsecure functions
						\item Input validation and overflow protection
					\end{itemize}
			\end{itemize}
		\subsubsection {Secure Code Usage}
			\begin{itemize}
				\item Code reuse -- using a block of code in a different context
				\item Third party libraries/DLLs
				\item Software Development Kit (SDKs)
				\item Stored procedures
			\end{itemize}
		\subsubsection {Other Secure Coding Practices}
			\begin{itemize}
				\item Unreachable and Dead code -- code that does not affect program flow
				\item Obfuscation/camouflage -- disguise nature of code, inhibit
					reverse engineering
			\end{itemize}
		\subsubsection {Static Code Analysis}
			\begin{itemize}
				\item Static/source code analysis
					\begin{itemize}
						\item Submit code for analysis by automated software
					\end{itemize}
				\item Manual Code review
					\begin{itemize}
						\item Human analysis of source code
					\end{itemize}
			\end{itemize}
		\subsubsection {Dynamic Code Analysis}
			\begin{itemize}
				\item Run application in a staging environment for testing
				\item Fuzzing and stress testing
					\begin{itemize}
						\item Application UI
						\item Protocol
						\item File format
					\end{itemize}
			\end{itemize}
	\subsection {Implement Secure Script Environments}
		\subsubsection {Scripting}
			\begin{itemize}
				\item Automation of activity through programs and scripts
				\item Basic elements of a script
					\begin{itemize}
						\item Parameters
						\item Branching and looping statements
						\item Validation and error handlers
						\item Unit Tests
					\end{itemize}
				\item Scripting Language
				\item Domain-specific
				\item Orchestration Tools
				\item Syntax
			\end{itemize}
		\subsubsection {Python Script Environment}
		\subsubsection {Execution Control}
			\begin{itemize}
				\item Prevent use of unauthorized code
				\item Allow lists and block lists
				\item Code signing
				\item OS-signing Execution Control
					\begin{itemize}
						\item Software Restriction Policies (SRP)
						\item AppLocker
						\item Windows Defender Application Control (WDAC)
						\item SELinux
						\item AppArmor
					\end{itemize}
			\end{itemize}
		\subsubsection {Malicious Code Indicator}
			\begin{itemize}
				\item Detection through monitoring platforms or host/process behavior
					analysis
				\item Shellcode -- create process/inject DLL
				\item Credential Dumping
				\item Lateral Movement/insider attack
				\item Persistence
			\end{itemize}
		\subsubsection {Man-in-the-Browser Attack}
			\begin{itemize}
				\item Compromise browser
				\item Malicious plug-in/DLL
				\item Browser Exploitation Framework (BEEF)
				\item Exploit kits
			\end{itemize}
	\subsection {Summarize Deployment and Automation Concepts}
		\subsubsection {Application Development, Deployment and Automation}
			\begin{itemize}
				\item DevSecOps
				\item Completion of tasks without human intervention
				\item Automation facilitates better scalability and elasticity
			\end{itemize}
		\subsubsection {Secure Application Development Environments}
			\begin{itemize}
				\item Software development life cycle (SDLC) -- waterfall and Agile
				\item Quality Assurance (QA)
				\item Development Environments
				\item Preserving environment integrity
					\begin{itemize}
						\item Sandboxing
						\item Secure baseline
						\item Integrity management
					\end{itemize}
			\end{itemize}
		\subsubsection {Provisioning, Deprovisioning and Version Control}
			\begin{itemize}
				\item Provisioning -- process of deploying an application to the target
					environment (installing/setup, instance)
				\item Deprovisioning -- process of removing an application from packages
					or instances
				\item Version Control
					\begin{itemize}
						\item Customer Version ID
						\item Developer Build ID
						\item Source code version control
						\item Code commits and backups
					\end{itemize}
			\end{itemize}
		\subsubsection {Automation/Scripting Release Paradigms}
			\begin{itemize}
				\item Waterfall and Agile SDLC
				\item Continuous integration
					\begin{itemize}
						\item Commit updates often
						\item Reduce commit conflicts
					\end{itemize}
				\item Continiuoos Delivery
					\begin{itemize}
						\item Push updates to staging infrastructure
					\end{itemize}
				\item Continuous Deployment
					\begin{itemize}
						\item Push updated code to production
					\end{itemize}
				\item Continuous monitoring and automated courses of action
				\item Continuous validation
			\end{itemize}
		\subsubsection {Software Diversity}
			\begin{itemize}
				\item Runtime environment -- Compiled/interpreted
				\item Software diversity as obfuscation
				\item Security by diversity -- avoid monocultures to make
					attacks harder to develop
			\end{itemize}

\section {Implementing Secure Cloud Solutions}
